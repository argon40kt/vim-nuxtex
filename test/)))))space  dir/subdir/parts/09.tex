\section{OPTIONS}

\begin{verbatim*}
b:nuxtex_makeprg				*b:nuxtex_makeprg*
	Type: string
	This value defines the compiler command to execute if the compiler
	plugin selected as nuxtex. This is similar as |g:nuxtex_makeprg|.

g:nuxtex_makeprg				*g:nuxtex_makeprg*

	Type: string
	This value defines the compiler command to execute if the compiler
	plugin selected as nuxtex. This parameter should be set before execute
	`:compiler nuxtex`.

	Example:
>
		let g:nuxtex_makeprg = "latexmk -pdfdvi -latex=latex -synctex=1 -e \"\\$dvipdf='dvipdfmx \\%O \\%S';\\$bibtex='bibtex';\" %:p"
<
	This example for using latexmk to compile `latex` command, convert dvi
	by `dvipdfmx` and use `bibtex`.
	See also |nuxtex-set-compiler|.

b:nuxtex_output_pdf				*b:nuxtex_output_pdf*
	Type: string
	This value is user manual configuration of  output pdf file generated
	from the (La)TeX source. This value is buffer variable, so it should
	be set on the buffer opened the (La)TeX source you would like to
	check the output.

	Example:
>
		let b:nuxtex_output_pdf = '/path/to/foo/bar.pdf'
<

g:nuxtex_force_quickfix				*g:nuxtex_force_quickfix*
	Type: boolean
	Default: |v:false|
	This value activate NuxTeX quickfix feature forcibly even if any
	compiler plugin selected. This plugin detect compiler plugin set for
	it for the buffer in the |QuickFixCmdPre| timing. So user or other
	plugins select other buffer in the timing, NuxTeX cannot detect the
	target buffer set as `:compiler nuxtex` in past.
	It is useful for in this case, but it is not recommended to set the
	parameter. Set `let g:nuxtex_force_quickfix = v:true` to activate
	this feature.

g:nuxtex_gz_parse				*g:nuxtex_gz_parse*
	Type: boolean
	Default: |v:true|
	This variable activate or deactivate *.synctex.gz file parse for
	Forward search feature. If this variable set as |v:true| and `gzip`
	command is installed in the $PATH, this plugin will analyze
	*.synctex.gz file when Forward search command is executed and choose
	correct *.synctex.gz file generated from the (La)TeX source. This is
	useful for there are multiple *.synctex.gz files are in the directory
	by some reasons.
	If there are any reasons, it is recommended to set no configuration
	for the parameter. If you would like to set this parameter as |v:true|
	but `gzip` command has not been installe in the $PATH, you can set
	|g:nuxtex_gzip_path|.
	See also |nuxtex-synctex|.

g:nuxtex_gzip_path				*g:nuxtex_gzip_path*
	Type: string
	It is possible to set the `gzip` command path by setting this variable.
	It is useful if |g:nuxtex_gz_parse| is set as v:true, but `gzip`
	command is not installed in the $PATH.

	Example:
>
	let g:nuxtex_gzip_path = '/usr/local/bin/gzip'
<

g:nuxtex_open_method				*g:nuxtex_open_method*
	Type: char
	Default: 't'
	This variable for configure the rule for manipulating the buffer not
	opened in any windows (but stored in the buffer) when the Backward
	search function executed.
	The default value is 't' and it means it will open the stored buffer
	in the new tab when the Backward feature select the buffer called from
	pdf viewer.
	This variable can be set as below.

't'	Open stored buffer in the new tab.

'h'	Split the currrent window and open stored buffer in the left hand side
	of the current window.

'j'	Vertical split the currrent window and open the stored buffer in the
	below of the current window.

'k'	Split the currrent window and open stored buffer in the right hand side
	of the current window.

'l'	Vertical split the current window and open the stored buffer in the
	above side of the current window.

'H'	Open the new window in the left side of the current tab and display
	the stored buffer.

'J'	Open the new window in the bottom of the current tab and display the
	stored buffer.

'K'	Open the new window in the top of the current tab and display the
	stored buffer.

'L'	Open the new window in the right side of the current tab and display
	the stored buffer.

'c'	Open the stored buffer in the current window.

g:nuxtex_python_cmd				*g:nuxtex_python_cmd*
	Type: string
	Default: 'python3'
	This variable is for to configure the path and execution file name
	of the python. The plugin use python and some libraries to execute
	SyncTeX feature. So it should be set as correct python command to
	execute the feature.

	Example:
>
	let g:nuxtex_python_cmd = '/usr/local/bin/python'
<

g:nuxtex_sys_enc				*g:nuxtex_sys_enc*
	Type: string
	This variable is used in analyzing *.synctex.gz file. It defines `gzip`
	command stdout |encoding|. NuxTeX will convert the stdout |encoding|
	from this variable to editor defined |&enc|. This feature stands on
	|iconv()|.

	Example:
>
	let g:nuxtex_sys_enc = "utf-8"
>

g:nuxtex_viewer_type				*g:nuxtex_viewer_type*
	Type: string
	Default: 'evince'
	This variable is used for select pdf viewer used in SyncTeX feature.
	It is possible to select from 'evince', 'atril', 'xreader' and
	'zathura'. If it is not selected, GNOME Document Viewer (Evince) will
	be used in the plugin default.
	See also |nuxtex-synctex|.

	Example:
>
	" Select pdf viewer as GNOME Document Viewer (Evince)
	let g:nuxtex_viewer_type = 'evince'

	" Select pdf viewer as Atril (Default viewer of MATE Desktop)
	let g:nuxtex_viewer_type = 'atril'

	" Select pdf viewer as Xreader (Default viewer of Linux Mint Cinnamon)
	let g:nuxtex_viewer_type = 'xreader'

	" Select pdf viewer as Zathura (It has Vim like keybind)
	let g:nuxtex_viewer_type = 'zathura'
>

g:nuxtex_zathura_cmd				*g:nuxtex_zathura_cmd*
	Type: string
	Default: 'zathura'
	This valiable defines the path and execution file name of the zathura.
	It is useful for if you would like to use zathura not installed in the
	\$PATH. It is only activated if |g:nuxtex_viewer_type| is set as
	'zathura'.

	Example:
>
	let g:nuxtex_zathura_cmd = '/usr/local/bin/zathura'
<

g:nuxtex_zathura_opt				*g:nuxtex_zathura_opt*
	Type: string
	Default: '--synctex-forward "@line:@col:@tex" "@pdf"'
	This variable defines the zathura command option for Forward search
	feature. The below strings will be substituted in the command.

- '@line' will be replaced in the current buffer line number.
- '@col' will be replaced in the current buffer column number.
- '@tex' will be replaced in the current buffer file name.
- '@pdf' will be replaced in the (La)TeX output pdf file name found in the
         logic described in |nuxtex-synctex|.

	For example, in case of |g:nuxtex_zathura_cmd| set as 'zathura', this
	variable set as '--synctex-forward "@line:@col:@tex" "@pdf"', the
	cursor position is 100 line and 5 column of the 'foo.tex' and the
	output pdf file is '/path/to/foo.pdf', the plugin will execute command
	as below.
>
	zathura --synctex-forward "100:5:foo.tex" "/path/to/foo.pdf"
<

	This variable is used in if |g:nuxtex_viewer_type| set as 'zathura'.
\end{verbatim*}

