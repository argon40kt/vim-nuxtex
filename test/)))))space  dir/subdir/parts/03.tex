%!TeXroot=../../output/Makefile
\section{USAGE}
%  !  TeX  root = ../../output/Makefile

1. Install this plugin and dependencies (if it is neccesarry).
   See |nuxtex-dependencies|.

2. Set up the document compilation method. See detail in |nuxtex-set-compiler|.

3. Configure \verb|g:nuxtex_viewer_type| as the pdf viewer which you would like to use. The value can be chosed from 'evince', 'atril', 'xreader' and 'zathura'. See \verb|nuxtex-synctex|.

4. Open the LaTeX source and edit it. Once you would like to compile the source, first, choose the compiler plugin as NuxTeX by `:compiler nuxtex`.
   After that, |:make| to compile. If you don't have any error, go next step.
   Otherwise, do |:copen| and check the error occured line and fix the root cause. Then, try compile again while there are no error.

5. After compilatoin succeed, you can jump both from source to pdf and pdf to source. To jump to pdf, just |<localleader><localleader>nf| on the LaTeX source. To jump from pdf to source, Ctrl-<Left Click> on the point you would like to see the source on the viewer.

